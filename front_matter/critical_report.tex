\documentclass[shorttitlesize=55]{ees}

\begin{document}

\eesTitlePage

\eesCriticalReport{
  – & –   & –     & \textit{Sanctus}, \textit{Osanna},
                    \textit{Agnus Dei (1)}, and \textit{Agnus Dei (3)} have
                    been written by Jan Dismas Zelenka. The title page of \B1
                    also mentions a \textit{Kyrie} by Zelenka,
                    but the latter movement presumably has been lost. \\
  \midrule
  1 & 77  & org   & 4th \eighthNote\ in \B1: A8 \\
    & 103–215 & – & According to Zelenka’s notes in \B1, the
                    \textit{Domine Fili} and \textit{Domine Deus, Agnus Dei}
                    should be omitted (presumably since they are later reused
                    as \textit{Agnus Dei (2)} and \textit{Benedictus},
                    respectively). Nevertheless, both movements are reproduced
                    in this edition. \\
    & 152 & clno  & 2nd \eighthNote\ in \B1: f″8 \\
    & 193 & T     & last \sixteenthNote\ in \B1: b16 \\
    & 194 & T     & 6th \sixteenthNote\ in \B1: c′16 \\
    & 212 & vla 2 & 8th to last \sixteenthNote\ in \B1: 5 × f′16 \\
    & 236 & vla 1 & 3rd \halfNote\ in \B1: f′4.–g′8 \\
    & 266–363 & vl & The directives “Oboe”, “T.” and “Vv.” in vl 2
                    indicate the beginning and end of segments where the oboe
                    should play instead of the violin, unison with the violin,
                    or pause, respectively. Based on these directives,
                    the oboe part of this edition has been assembled.
                    Nevertheless, the directives are retained in vl 2.
                    Similar directives for a tromba appear in vl 1, i.\,e.,
                    “Tromba”, “T.”, and “Vv.”. However, a separate tromba part
                    has been written in \B1. These two alternative tromba parts
                    are here shown as “tr” and “clno” parts, respectively.
                    In tr, bars 266, 277, 282–284, 305–309, 323–325, 337,
                    and 360–362 were emended to accommodate the
                    instrument’s range. \\
    & 341 & vl 2  & 2nd/3rd \quarterNote\ in \B1: \flat b′8.–a′16–\flat b′4 \\
    & 382 & vl 1  & last \quarterNote\ in \B1: e″8–f″8 \\
    & 383 & S     & last \quarterNote\ in \B1: f″8–g″8 \\
    & 385–387 & T & in \B1 unison with B \\
    & 386 & vla 1 & 4th \eighthNote\ in \B1: d″8 \\
    & 387 & A     & rhythm emended \\
    & 401 & vl 1  & 3rd \quarterNote\ in \B1: c″4 \\
    & 401 & vl 2  & 3rd \quarterNote\ in \B1: a′4 \\
  \midrule
  2 & 1–6 & vla 2 & in \B1 written in alto clef despite a tenor clef
                    in front of the staff \\
    & 17–65 & –   & The \textit{Benedictus} is a parody of the
                    \textit{Domine Deus, Agnus Dei}. See the latter movement
                    for comments to bars 43 (T → 193), 44 (T → 194),
                    and 62 (vla 2 → 212). \\
    & 76  & vl 2  & 6th \eighthNote\ in \B1: g″8 \\
    & 79  & vla 2 & 6th \eighthNote\ in \B1: e′8 \\
  \midrule
  3 & 12–75 & –   & The \textit{Agnus Dei (2)} is a parody of the
                    \textit{Domine Fili}. See the latter movement
                    for the comment to bar 61 (clno → 152). \\
    & 81  & vla 2 & 1st \halfNote\ in \B1: f′4–f′4 \\
    & 84–118 & –  & The \textit{Dona nobis} presumably is a parody of the
                    \textit{Amen} in \textit{Cum Sancto Spiritu}, although
                    there are neither remarks nor alternative lyrics in the
                    \textit{Amen}. See the latter movement for comments
                    to bars 92 (vl 1 → 382), 93 (S → 383), 95 (T → 385),
                    96 (vla 1 → 386), 97 (A → 387), and 111 (vl 1/2 → 401). \\
}

\eesToc{}

\eesScore

\end{document}
